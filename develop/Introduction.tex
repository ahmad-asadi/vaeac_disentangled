\section{Introduction}
High-frequency trading (HFT) models have revolutionized financial markets, becoming a cornerstone of modern stock and cryptocurrency trading. While traditional HFT focuses on capitalizing on millisecond-level market inefficiencies, a growing body of research highlights the importance of developing models that operate on longer but still high-resolution time frames, such as hourly intervals. In volatile markets like cryptocurrencies, hourly price movements capture significant trends and reversals, providing opportunities for strategic portfolio adjustments. By leveraging advanced algorithms, real-time analysis, and frequent decision-making, models operating at this granularity can effectively identify short-term price patterns, optimize asset allocation, and enhance portfolio performance. The success of such strategies lies in their ability to respond quickly to evolving market conditions, maximizing returns while mitigating risk over frequent but manageable trading intervals.

A critical aspect of HFT strategies is the prevention of small, cumulative losses, which can have an outsized impact on long-term portfolio performance. Unlike traditional trading, where losses can be balanced over extended periods, HFT operates on razor-thin margins and high transaction volumes, making small losses particularly detrimental. When left unchecked, these losses can compound rapidly, eroding gains and leading to substantial underperformance. Conversely, mitigating minor losses through precise, data-driven decision-making can result in exponential returns over time. This dynamic underscores the importance of developing models capable of analyzing micro-level price behaviors and reacting swiftly to unfavorable conditions.

High-frequency trading techniques rely heavily on integrating diverse information sources to identify and capitalize on market trends. Advanced models aggregate technical indicators (\citet{chen2018profitability}), quantitative financial data (\citet{gomber2015high}), crowd-sourced sentiment analysis \citet{liu2023multi}, and multi-source data fusion approaches (\citet{asadi4423354multi, liu2023multi}). By combining these inputs, HFT models achieve a comprehensive understanding of market behavior, improving their precision in trend detection and risk mitigation (\citet{li2014online}). This multi-faceted approach empowers traders to not only avoid minor losses but also uncover critical pivot points in stock price series—key moments where trends shift direction. Identifying these pivot points before they occur is central to developing robust strategies capable of maximizing returns while navigating the market's inherent volatility and uncertainty.

While existing research emphasizes the importance of fusing data from multiple sources to identify price series pivot points, we argue that a deeper understanding of price behavior is necessary for predicting trend changes. The ability to anticipate these shifts requires disentangling the factors influencing price movements, such as trend direction and volatility, to predict the probability of changes in the trend of each asset's price in the near future. In highly dynamic markets like cryptocurrencies, even short-term price reversals can significantly impact trading strategies, making it essential to accurately forecast these changes to optimize portfolio performance.

A key innovation of this study is the introduction of a coin-switching strategy that challenges the conventional diversification approach typically employed in portfolio management. Diversification, which involves allocating investments across multiple assets (\citet{markovitz1959portfolio}), is widely used to mitigate risk by reducing dependence on any single asset's performance. However, in high-frequency trading (HFT), where decisions are made at fine-grained time intervals, this traditional approach may not fully exploit opportunities arising from short-term price movements.

In our proposed model, the entire investment value is allocated to a single coin at the start of each investment period, based on the predicted probability of trend shifts for all available coins. This focused strategy enables the model to dynamically select the coin most likely to exhibit a favorable trend, effectively reducing risk while maximizing returns. By disentangling trends and variances for each coin, the model ensures that the selection is data-driven and reflective of the near-term market conditions. Over multiple investment periods, this coin-switching strategy behaves similarly to the cumulative effects of diversification, as the portfolio iteratively adapts to the market's most promising opportunities. Moreover, this approach parallels the concept of concurrent task execution in CPU processing, where tasks are executed in rapid succession to achieve results comparable to parallel computing.

In this paper, we propose a novel framework for stock portfolio management that focuses on predicting the probability of near-future trend changes for each asset. The contributions of this study are as follows:

\begin{enumerate}
	\item We introduce a model that disentangles price behavior into meaningful components to predict the likelihood of trend shifts in each asset's price.
	\item We propose a deep reinforcement learning (DRL) model capable of dynamically switching between assets based on the predicted probabilities of trend changes.
	\item We propose a coin-switching strategy for high-frequency trading, where the entire investment is allocated to a single asset during each period. This approach, informed by disentangled trend and variance features, reduces risk while optimizing returns by focusing on the most promising asset at each interval.
	\item The proposed coin-switching method mirrors the cumulative effects of diversification over time by dynamically reallocating investments across assets, akin to concurrent task execution in CPUs.
	\item We demonstrate that the proposed model outperforms state-of-the-art portfolio management strategies in the cryptocurrency market, achieving superior returns and improved portfolio stability.
\end{enumerate}

By leveraging disentangled price features, reinforcement learning, and innovative allocation strategies, the proposed framework addresses the challenges of frequent decision-making in dynamic financial markets and enhances the ability to adapt to near-term price trends.

The rest of this paper is organized as follows: In Section 2 we review the existing models for stock portfolio management and stock price trend predictors. Section 3 describes in detail the structures of the time-series uncertainty estimation model and the deep reinforcement learning model. The proposed model is then evaluated and the evaluation results are reported and discussed in Section 4. Finally, the conclusions of the paper are summarized in Section 5.