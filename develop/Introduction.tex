\section{Introduction}
Stock portfolio management is a crucial aspect of investment strategy, involving the selection and allocation of assets to achieve specific financial goals. The importance of portfolio management lies in its ability to optimize risk and return trade-offs, ultimately leading to the maximization of wealth for investors. Investors typically mitigate risk and enhance returns by carefully selecting a diverse range of assets.

Portfolio diversification is a critical strategy in risk management within investment portfolios. This approach involves allocating investments across various asset classes and industries to mitigate the impact of individual asset price fluctuations on the overall portfolio performance. By diversifying, investors can reduce volatility and safeguard against substantial losses in any single investment, thus enhancing the stability and robustness of the portfolio's returns. Additionally, a well-diversified portfolio has the potential to enhance risk-adjusted returns by leveraging the advantages of diverse market conditions (\citet{markovitz1959portfolio}).

High frequency trading techniques have emerged as a prominent strategy in financial markets, leveraging advanced algorithms and rapid execution speeds to capitalize on fleeting market opportunities (\citet{li2014online}). The inherent volatility and speed of high frequency trading present unique challenges, particularly in avoiding small losses that can quickly accumulate. 

In the realm of high frequency trading, to avoiding small losses often strategies rely on the aggregation of information from diverse sources (\citet{thakkar2021fusion}). By incorporating a wide range of technical indicators (\citet{chen2018profitability}), crowd-sourced data (\citet{liu2023multi}), quantitative data (\citet{gomber2015high}), and multi-type data fusion models (\citet{liu2023multi}) high frequency traders can enhance their ability to identify and react to market trends with greater precision and speed. This multi-faceted approach not only enables traders to more effectively manage risk by mitigating small losses but also provides a comprehensive view of market dynamics that can inform more strategic trading decisions. By leveraging a sophisticated blend of quantitative analysis and real-time information sources, high frequency trading techniques can strive to optimize performance and navigate the complex interplay between risk and return in dynamic financial markets.
%In contrast, traditional portfolio management approaches such as the Markowitz trade-offs between risk and return offer a more measured and systematic approach to balancing investment decisions. This paper aims to critically compare the effectiveness of high frequency trading techniques with the Markowitz framework in navigating the complex dynamics of risk and return in financial markets, emphasizing the importance of mitigating small losses in high frequency trading strategies.

The majority of the researchers active in portfolio management believe that employing complex models for fusing the information from different sources enables models to find pivot points in stock price series(\cite{thakkar2021fusion}). However, we believe that for finding pivot points in stock price series before occurrence it is required to extract features of the price behavior by disentangling the impact of interconnected factors from the last time steps and measuring the uncertainty of the persistence of the current trend in price series of each stock. Due to the unpredictable nature of financial markets and the dynamic interactions between various factors, uncertainty is inherent in stock portfolio management. Then, measuring and managing uncertainty is essential for making investment decisions and designing robust portfolio strategies. 

In this paper, a two-step framework is presented which is able to first extract the features indicating the probability of price trend change in each stock in the future, and second, to adjust the portfolio based on the extracted features in order to minimize the loss probability at each transaction. The contributions of this work can be concluded as follows:


\begin{enumerate}
	\item A feature extraction module is presented that is able to extract features indicating the probability of the price trand change in the future for each stock.
	\item A deep reinforcement learning model with a novel reward function is proposed which is able to switch between the stocks in the market based on the uncertainty of their current price trend in the future.
	\item The proposed model outperforms the state-of-the-art models in stock portfolio management in crypto-currencies market.
\end{enumerate}

The rest of this paper is organized as follows: In Section 2 we review the existing models for stock portfolio management and stock price trend predictors. Section 3 describes in detail the structures of the time-series uncertainty estimation model and the deep reinforcement learning model. The proposed model is then evaluated and the evaluation results are reported and discussed in Section 4. Finally, the conclusions of the paper are summarized in Section 5.