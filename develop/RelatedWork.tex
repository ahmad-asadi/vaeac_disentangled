\section{Related Work}

The application of reinforcement learning (RL) techniques, particularly deep reinforcement learning (DRL), in portfolio management has gained traction in recent years. DRL models are able to learn optimal trading policies through interactions with the market, adjusting asset allocations based on real-time data and the evolving market environment. These models can dynamically adjust portfolio weights by considering a wide range of factors, including price trends, volatility, and market sentiment. However, despite their potential, DRL models that rely on diversification may not be well-suited to handle the challenges posed by high-frequency trading in volatile markets like cryptocurrencies, where small fluctuations can quickly lead to losses.


Deep learning models have become an essential tool in stock portfolio management due to their ability to process vast amounts of data and identify complex patterns that traditional models often miss. Traditional portfolio management approaches typically rely on statistical models (\citet{li2012pamr}) or optimization techniques (\citet{pennanen2012introduction}) to guide investment decisions. With the rise of deep learning and deep reinforcement learning (DRL) methods, more sophisticated, data-driven strategies have emerged, often incorporating technical indicators for better portfolio management (\citet{ayala2021technical}, \citet{agrawal2022stock}, \citet{taghian2021reinforcement}, \citet{taghian2022learning}).

In financial markets, the ability of deep learning models to handle complex, multi-modal data has led to the adoption of information fusion techniques to improve portfolio management. One approach involves integrating multiple types of data, such as historical prices, financial statements, and social media sentiment, into a unified model, allowing for a more comprehensive market analysis (\citet{asadi4423354multi}). Additionally, ensemble methods, which combine outputs from multiple models trained on different data subsets or architectures, have been shown to enhance prediction accuracy and robustness (\citet{carta2021multi}). Other advancements include the use of attention mechanisms to focus on relevant information across markets (\citet{zhao2022stock}) and the application of transformer models for fusing temporal data, which helps in optimizing portfolio performance by minimizing small losses (\citet{gullotto2021portfolio}, \citet{kisiel2022portfolio}). Finally, recent work by \citet{abdulsahib2024cross} introduces disentangled representation learning to capture shared features between markets, providing a more nuanced understanding of market interdependencies and improving prediction accuracy.

The use of deep reinforcement learning (DRL) models in high-frequency trading (HFT) is expanding rapidly. \citet{asare2024deep} introduced a CNN-based model designed for generating 15-minute buy/sell signals for cryptocurrencies. \citet{sun2022deepscalper} developed DeepScalper, a model for intraday trading that combines a dueling Q-network to handle large action spaces and an encoder-decoder framework to extract multi-resolution temporal data. \citet{qin2024earnhft} proposed the EarnHFT model, a hierarchical RL framework that calculates optimal action-values using dynamic programming as a Q-teacher, creates a pool of diverse RL agents for different market trends, and employs a minute-level router to select an RL agent from the pool. \citet{zong2024macrohft} introduced MacroHFT, another hierarchical RL model that trains a router to choose agents from the pool, incorporating a memory-augmented, context-aware RL model to address agent biases. \citet{fatemi2024finvision} proposed a multi-modal, multi-agent system where specialized LLM-based agents process diverse financial data, such as news reports, candlestick charts, and trading signals.

Another approach in high-frequency trading involves predicting future prices and selecting portfolios based on these predictions. Despite challenges like high volatility and non-stationary data, deep learning and feature selection techniques show promise in cryptocurrency forecasting (\citet{otabek2024prediction}). \citet{akyildirim2021prediction} explores the predictability of twelve cryptocurrencies using SVM-based machine learning models, achieving the best and most consistent results. \citet{ye2021predicting} uses ANNs to predict price directions for Bitcoin, Ethereum, and Cardano, effectively capturing complex patterns. \citet{liu2021forecasting} employs stacked denoising autoencoders for Bitcoin price prediction, managing both direction and level predictions. \citet{jay2020stochastic} integrates stochastic processes with neural networks to improve prediction accuracy.

While much of the related work has concentrated on integrating information from various sources for stock portfolio management, this study takes a different approach by focusing on the fundamental factors that drive price behavior. The objective is to identify early signs of trend shifts and anticipate price pivot points before they materialize. To achieve this, we introduce a model that disentangles price behavior into meaningful components, predicting the likelihood of trend changes for each asset. We propose an Actor-Critic model that dynamically switches between assets based on the predicted probabilities of trend changes, optimizing returns while managing risks. Additionally, we present a coin-switching strategy tailored for high-frequency trading, where investments are concentrated in a single asset during each period. This strategy, informed by disentangled trend and variance features, aims to reduce risk and maximize returns by focusing on the most promising asset. Furthermore, the coin-switching method effectively mirrors the cumulative benefits of diversification over time by reallocating investments across assets.