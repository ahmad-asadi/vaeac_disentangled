\section{Related Work}
In general, portfolio management studies can be categorized into the following closely related areas: Portfolio Optimization, Portfolio Allocation, and Portfolio Selection \cite{ozbayoglu2020deep}. Meanwhile, asset pricing models that aim to predict the future prices of stocks and derivatives in financial markets are also common approaches that help portfolio management models to propose portfolio vectors based on the predicted prices \cite{nazareth2023financial}.
\subsection{Information Fusion in Stock Portfolio Management}

The use of deep learning models in stock portfolio management is very well justified due to their ability to process vast amounts of data and extract complex patterns that traditional models may overlook. Traditional portfolio management strategies often rely on statistical models (\citet{li2012pamr}) and optimization models (\citet{pennanen2012introduction}) to make investment decisions. However, with the advent of deep learning and deep reinforcement learning techniques, researchers and practitioners have started exploring more sophisticated and data-driven approaches mostly based on using technical indicators to portfolio management (\citet{ayala2021technical}, \citet{agrawal2022stock}, \citet{taghian2021reinforcement}, and \citet{taghian2022learning}).

Due to the capabilities of deep learning models in financial markets specially their abilities to process large amounts of complex data, the fusion of various types of information plays a crucial role in enhancing the performance of deep learning-based portfolio management systems. One common approach to information fusion in deep learning-based stock portfolio management is the use of multi-modal data inputs (\citet{asadi4423354multi}). This involves combining different types of data, such as historical price data, financial statements, news sentiment, and social media feeds, into a single model. By incorporating multiple modalities of information, these models can capture a more comprehensive view of market dynamics and make more informed investment decisions.

Another approach to information fusion is the use of ensemble methods, where multiple deep learning models are trained on different subsets of data or with different architectures (\citet{carta2021multi}). The outputs of these models are then combined to generate a consensus prediction, which can be more robust and accurate than any individual model. \citet{lin2022multiagent} proposed a deep reinforcement learning (DRL) based model with multiple agents and designed a long-term reward function to reduce the risk of investment with fusing the decisions of different agents. \citet{hao2023stock} proposed a three-dimensional fuzzy representation of stock price trend and employed an ensemble DRL model for stock portfolio management based on the fuzzy representation of the price trend.

Furthermore, attention mechanisms have been proposed as a way to selectively focus on relevant information within a dataset, specifically in gathering information from multiple markets (\citet{zhao2022stock}). Another important deep learning model which is used for fusing temporal information for stock portfolio management is the Transformers model (\citet{gullotto2021portfolio}). \citet{kisiel2022portfolio} proposed a deep learning model based on a transformer for minimizing the small losses of stock trading with optimizing the Sharpe ratio \citet{sharpe1998sharpe} directly. \citet{liu2023revolutionising} combined a non-stationary transformer model with a DRL model for fusing macro-economic information with targeted news sentiments.

%Overall, the survey of different approaches to information fusion in deep learning-based stock portfolio management highlights the importance of integrating diverse sources of data and leveraging advanced techniques to enhance performance and robustness. By combining multiple modalities, using ensemble methods, and incorporating attention mechanisms, researchers and practitioners can develop more sophisticated and effective models for managing stock portfolios in dynamic financial markets.
%
%
%Deep reinforcement learning (DRL) is another area of interest in stock portfolio management, as it enables the development of dynamic trading strategies that adapt to changing market conditions. DRL algorithms, such as the Deep Q-Network (DQN) and Proximal Policy Optimization (PPO), have been utilized to train agents to make buy/sell decisions based on market signals and portfolio performance metrics. These models can learn optimal trading policies through trial-and-error interactions with the market environment, leading to potentially more robust and adaptive portfolio management strategies.


\subsection{Uncertainty Measurements in Stock Trading Strategies}

Uncertainty is a fundamental aspect of financial markets, and accurately measuring and managing uncertainty is crucial for developing effective stock trading strategies. In recent years, researchers have focused on exploring various methods and metrics to quantify uncertainty in stock trading, with the aim of improving decision-making processes and reducing risk exposure. This literature review provides an overview of the key studies and approaches related to uncertainty measurements in stock trading strategies (\citet{abdar2021review}).


The first category of models proposed for portfolio management with considering the risk of investment, are those that try to minimize investment risk by diversification of the proposed portfolios. \citet{du2022mean} proposed a mean-variance portfolio proposition based on the co-integrated stocks and their correlation. 

Considering risk-related measures for optimizing deep learning model weights is another approach in which the trained model would consider the uncertainty of the stock prices in portfolio proposition. \citet{syu2020portfolio} proposed a DRL based model in which the Sharpe ratio of portfolio proposals is used to optimize the model weights.

A novel model recently introduced by \citet{abdulsahib2024cross} in this field utilizes disentangled representation learning to break down the input features from various markets and identify the common factors during the representation learning phase. By decomposing the features of distinct markets and identifying shared features between them, the model can better understand the interconnected dynamics between the markets and predict price movements in one market based on the behavior of shared features with another market. Furthermore, the concept of disentangled representation learning was previously employed by \citet{abdulsahib2023glad} to study the price dynamics of stocks in financial markets.


While most of related work has focused on integrating information from various sources in stock portfolio management, this study seeks to break down the fundamental factors that influence price behavior. The goal is to detect indications of trend changes and identify price pivot points before they occur. To achieve this, disentangled representation learning is utilized to create a latent space based on historical stock price data, where latent features indicate potential price changes. A feature extractor is developed to detect price pivot points, and an actor-critic model is proposed to pinpoint the optimal time to switch between stocks to mitigate losses.
