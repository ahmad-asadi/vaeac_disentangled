\section{Conclusion}


In conclusion, the proposed model for stock portfolio management in the cryptocurrencies market demonstrates several key advantages and some limitations that are important to consider:

\subsection{Advantages}

\begin{enumerate}
	\item Effective uncertainty estimation: The integration of a Variational Autoencoder (VAE) enables our model to accurately quantify uncertainty in stock price predictions, providing valuable insights for risk management and decision-making.
	
	\item Dynamic portfolio optimization: The Actor-Critic neural network architecture allows for adaptive and dynamic portfolio rebalancing based on changing market conditions, leading to improved performance and resilience against volatility.
	
%	\item Robust Performance: Through extensive experimentation and comparison with state-of-the-art models, our method consistently achieves competitive results in terms of return on investment, Sharpe ratio, and maximum drawdown, showcasing its effectiveness in generating optimal stock portfolios.
	
	\item Real-world applicability: Leveraging a comprehensive dataset of historical cryptocurrency prices, our model operates under realistic market conditions, enhancing its practical relevance and applicability for financial institutions and investors.
	
\end{enumerate}


\subsection{Weaknesses}

\begin{enumerate}
%	\item Complexity: The complexity of the proposed model, particularly the integration of multiple neural network components and hyperparameters, may pose challenges in implementation and interpretation for users without a deep understanding of machine learning techniques.
	
	\item Data dependency: The performance of our model heavily relies on the quality and availability of historical asset price data during the training phase, which may limit its effectiveness in scenarios where data is scarce or unreliable.
	
%	\item Training Time: Training neural networks for portfolio optimization can be computationally intensive and time-consuming, especially when dealing with large datasets and complex architectures, potentially hindering real-time decision-making processes.
	
\end{enumerate}


Overall, the benefits of our proposed model outweigh its limitations, as it offers a data-driven approach to assess uncertainty of price trends in stock portfolio management in the volatile cryptocurrencies market. By addressing uncertainties, optimizing portfolios dynamically our method provides a valuable tool for investors seeking to navigate the complexities of cryptocurrency trading with enhanced confidence and efficiency. Further research and refinement of the model could help mitigate its limitations and unlock even greater potential for AI-driven financial strategies in the future.